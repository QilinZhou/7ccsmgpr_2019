\section{Tools/Technology}
We will use the following tools and technology for the corresponding purposes:
\begin{itemize}

\item \textbf{Amazon S3}: We use S3 as our cloud/server endpoint. S3 provides APIs for Javascript and Java which we will exploit for our desktop and mobile client respectively. S3 handles encryption and notifies us of any potential vulnerabilities.

\item \textbf{Electron/Javascript}: For building the desktop application. It uses \emph{Chromium} and \emph{Node.js} on the backend and helps convert web application code (HTML/CSS) into desktop app.

\item \textbf{Android Studio}: For building mobile client.

\item \textbf{MySQL}: For Database. Note: We have not investigated other possibilities in depth at the moment and thus this choice is open for changes.

\item \textbf{Charles}: For web debugging. This will help us debug server calls as well as help simulate conflict scenarios.

\item \textbf{Mocha \& JUnit}: For writing unit tests on Javascript and Java respectively.

\item \textbf{GitHub + Git Bash/terminal}: Version control and code collaboration. We have created an organisation called \emph{deadlinefighters} of which everyone is a part of. Everyone has write access whereas \emph{Sivaranjani} and \emph{Letao Wang} have administrative privileges.

\item \textbf{Atom}: We use Atom as our code/text editor as it has Git and {\LaTeX} compatibility, giving us a single view of our important tools.

\item \textbf{{\LaTeX}}: We use {\LaTeX} for documentation and our {\LaTeX} file is also collaborated on GitHub. Every team member has {\LaTeX} compiler installed in their system along with a plugin for {\LaTeX} on Atom which helps compile while on the text editor.

\item \textbf{MockingBot}: MockingBot is used for the designing user interface of our desktop and mobile client. This will help streamline developer efforts.

\end{itemize}
